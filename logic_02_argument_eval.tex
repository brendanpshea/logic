\documentclass{beamer}
\usepackage{tikz}
\usepackage{booktabs}
\usepackage{array}

\usetheme{Madrid}
\usecolortheme{beaver}

\title{Evaluating Arguments}
\subtitle{Deductive, Inductive, and Abductive Reasoning}
\author{Brendan Shea, PhD}
\date{Intro to Logic}

\begin{document}
	
	\frame{\titlepage}
	
	% Slide 1
	\begin{frame}{Three Types of Arguments and How to Evaluate Them}
		\begin{itemize}
			\item Every day we encounter arguments that try to convince us of various conclusions, from scientific theories to criminal investigations.
			\item Not all arguments are created equal—some provide strong support for their conclusions while others are logically flawed.
			\item There are three main types of arguments: \textbf{deductive}, \textbf{inductive}, and \textbf{abductive}, each with different standards for evaluation.
			\item Understanding these differences is crucial for thinking critically about evidence and reasoning in any field.
		\end{itemize}
		
		\begin{alertblock}{Key Point}
			Different types of arguments require different evaluation criteria—what makes a deductive argument good is not the same as what makes an inductive argument good.
		\end{alertblock}
	\end{frame}
	
	% Slide 2
	\begin{frame}{The Detective's Toolkit: Why Different Arguments Need Different Standards}
		\begin{itemize}
			\item Think of a detective investigating a crime—they use multiple types of reasoning to build their case.
			\item Sometimes they use \textbf{deductive reasoning}: "If the suspect was at the coffee shop at 3 PM, then he couldn't have committed the murder at 3 PM across town."
			\item Other times they use \textbf{inductive reasoning}: "This suspect has committed similar crimes before, so he's likely our perpetrator."
			\item They also use \textbf{abductive reasoning}: "The best explanation for this evidence is that the butler did it."
		\end{itemize}
		
		\begin{example}
			A forensic scientist might use math (deduction) to calculate height in cm (based on height in inches), induce that the crime rate will increase next month based on trends (probable), or abduce that the victim knew their attacker based on the evidence (best explanation).
		\end{example}
	\end{frame}
	
	% Slide 3
	\begin{frame}{Deductive Arguments: When Conclusions Follow Necessarily}
		\begin{itemize}
			\item A \textbf{deductive argument} is one where the conclusion is supposed to follow necessarily from the premises.
			\item If the premises are true, then the conclusion \textit{must} be true—there's no possibility of the premises being true and the conclusion false.
			\item We evaluate deductive arguments as either \textbf{valid} (the conclusion does follow necessarily) or \textbf{invalid} (it doesn't).
			\item The truth of the premises is a separate question from whether the argument is valid.
		\end{itemize}
		
		\begin{block}{Definition: Deductive Argument}
			An argument where the conclusion is intended to follow with logical necessity from the premises. If valid and the premises are true, the conclusion must be true.
		\end{block}
	\end{frame}
	
	% Slide 4
	\begin{frame}{Valid vs. Invalid: The Foundation of Deductive Reasoning}
		\begin{itemize}
			\item A \textbf{valid} deductive argument is one where if all the premises were true, the conclusion would have to be true.
			\item An \textbf{invalid} deductive argument is one where the premises could be true but the conclusion could still be false.
			\item Validity is about the logical structure of the argument, not whether the premises are actually true in reality.
			\item A valid argument with true premises is called \textbf{sound}—this gives us the strongest possible support for a conclusion.
		\end{itemize}
		
		\begin{example}
			\textbf{Valid:} All detectives carry badges. Holmes is a detective. Therefore, Holmes carries a badge.
			\\[0.5em]
			\textbf{Invalid:} Most detectives are observant. Holmes is observant. Therefore, Holmes is a detective.
		\end{example}
	\end{frame}
	
	% Slide 5
	\begin{frame}{Common Deductive Argument Forms Table}
		\begin{itemize}
			\item Deductive arguments often follow recognizable patterns or \textbf{forms} that guarantee validity when used correctly.
			\item Learning these common forms helps you quickly identify valid deductive reasoning in everyday situations.
			\item Each form has a specific structure that, when followed precisely, ensures the conclusion follows necessarily from the premises.
			\item These forms appear frequently in legal reasoning, mathematical proofs, and logical problem-solving.
		\end{itemize}
		
		\begin{table}[h]
			\centering
			\scriptsize
			\begin{tabular}{|l|l|}
				\hline
				\textbf{Argument Form} & \textbf{Structure} \\
				\hline
				Categorical Syllogism & All A are B; All B are C; Therefore, All A are C \\
				\hline
				Modus Ponens (MP) & If P then Q; P; Therefore, Q \\
				\hline
				Modus Tollens (MT) & If P then Q; Not Q; Therefore, Not P \\
				\hline
				Hypothetical Syllogism (HS) & If P then Q; If Q then R; Therefore, If P then R \\
				\hline
				Mathematical Arguments & Based on definitions, axioms, and proven theorems \\
				\hline
			\end{tabular}
		\end{table}
	\end{frame}
	
	% Slide 6
	\begin{frame}{Sherlock Holmes and the Categorical Syllogism}
		\begin{itemize}
			\item A \textbf{categorical syllogism} involves three categories and shows how they relate to each other through two premises.
			\item The classic form is: All A are B, All B are C, Therefore All A are C—this structure guarantees validity.
			\item The key is ensuring that the middle term (B) properly connects the other two categories.
		\end{itemize}
		
		\begin{example}
			\textbf{Holmes's reasoning:}
			\begin{itemize}
				\item All murderers in this case left muddy footprints in the library.
				\item All people who left muddy footprints in the library entered through the garden door.
				\item Therefore, all murderers in this case entered through the garden door.
			\end{itemize}
		\end{example}
	\end{frame}
	
	% Slide 7
	\begin{frame}{Modus Ponens: Detective Rodriguez's Logical Breakthrough}
		\begin{itemize}
			\item \textbf{Modus Ponens} (MP) is the most fundamental deductive argument form: If P then Q, P, Therefore Q.
			\item This form captures the basic logic of conditional reasoning—when we know a condition leads to a result, and the condition occurs, we can deduce the result.
			\item The argument is valid regardless of what specific propositions we substitute for P and Q.
		\end{itemize}
		
		\begin{example}
			\scriptsize
			\textbf{Rodriguez's breakthrough:}
			\begin{itemize}
				\item If the suspect used his credit card after 6 PM, then he wasn't at the crime scene at 6 PM.
				\item The suspect used his credit card at 6:15 PM (bank records confirm this).
				\item Therefore, the suspect wasn't at the crime scene at 6 PM.
			\end{itemize}
		\end{example}
	\end{frame}
	
	% Slide 8
	\begin{frame}{Modus Tollens: How Inspector Chen Eliminated Suspects}
		\begin{itemize}
			\item \textbf{Modus Tollens} (MT) works by denying the consequent: If P then Q, Not Q, Therefore Not P.
			\item This form is particularly useful for elimination—when we know what should happen if something were true, but it doesn't happen.
			\item MT is the logical foundation of many ``proofs by contraction.''
		\end{itemize}
		
		\begin{example}
			\textbf{Chen's elimination process:}
			\begin{itemize}
				\item If Johnson committed the crime, then his DNA would be on the weapon.
				\item Johnson's DNA is not on the weapon (lab results show this).
				\item Therefore, Johnson did not commit the crime.
			\end{itemize}
		\end{example}
	\end{frame}
	
	% Slide 9
	\begin{frame}{Hypothetical Syllogisms: Following the Chain of Evidence}
		\begin{itemize}
			\item A \textbf{hypothetical syllogism} (HS) chains together conditional statements: If P then Q, If Q then R, Therefore If P then R.
			\item This form allows us to trace logical connections through multiple steps, linking distant causes to their effects.
			\item The strength of the chain depends on each individual conditional being true—one weak link breaks the whole argument.
		\end{itemize}
		
		\begin{example}
			\textbf{Chain of evidence:}
			\scriptsize
			\begin{itemize}
				\item If the victim knew the attacker, then there would be no signs of forced entry.
				\item If there were no signs of forced entry, then the crime was committed by someone with a key.
				\item Therefore, if the victim knew the attacker, then the crime was committed by someone with a key.
			\end{itemize}
		\end{example}
	\end{frame}
	

% Slide 10
\begin{frame}{Mathematical Arguments: CSI Thompson's Deductive Proof}
	\begin{itemize}
		\item \textbf{Mathematical arguments} use definitions, axioms, and previously proven theorems to reach conclusions with absolute certainty.
		\item These arguments are deductive because their conclusions follow necessarily from mathematical principles.
		\item The precision of mathematics makes these arguments particularly powerful because they provide certainty, not just probability. IF we start from true premises, THEN we get true conclusions.
	\end{itemize}
	
	\begin{example}
		\textbf{Thompson's deductive proof:}
		The evidence locker contains 247 pieces of evidence. Thompson removes 89 pieces for testing. By the mathematical definition of subtraction, there must be exactly 247 - 89 = 158 pieces remaining in the locker. This conclusion follows with absolute certainty from arithmetic.
	\end{example}
\end{frame}

	% Slide 11
	\begin{frame}{Testing Validity: The Counterexample Method}
		\begin{itemize}
			\item To test if a deductive argument is valid, we use the \textbf{counterexample method}: try to imagine a situation where all premises are true but the conclusion is false.
			\item If you can construct such a scenario, the argument is invalid; if you cannot, the argument is valid.
			\item This method works because validity means it's impossible for the premises to be true and the conclusion false.
			\item The counterexample doesn't need to reflect reality—it just needs to be logically possible.
		\end{itemize}
		
		\begin{block}{Testing Validity Step-by-Step}
			\begin{enumerate}
				\item Assume all premises are true
				\item Try to imagine the conclusion being false
				\item If this is possible, the argument is invalid
				\item If this is impossible, the argument is valid
			\end{enumerate}
		\end{block}
	\end{frame}
	
% Additional Slide: Worked-Out Counterexample
\begin{frame}{Worked-Out Counterexample: Testing Argument Validity}
	\begin{itemize}
		\item The counterexample method first requires \textbf{abstracting the argument form} by identifying the logical structure with variables.
		\item Once we have the abstract form, we \textbf{substitute new values} for the variables to create a new argument with the same structure.
		\item To show invalidity, we find substitutions that make the premises true but the conclusion false.
		\item If we can construct such a counterexample, we've proven the argument form is invalid.
	\end{itemize}
	
	\begin{block}{Complete Counterexample Process}
		\scriptsize
		\textbf{1. Detective Pikachu's Argument:} All criminals wear dark clothing at night. The suspect wore dark clothing last night. Therefore, the suspect is a criminal.
		\\[0.5em]
		\textbf{2. Abstract Form:} All A are B. X is B. Therefore, X is A.
		\\[0.5em]
		\textbf{3. Counterexample:} All cats are mammals. My son is a mammal. Therefore, my son is a cat.
	\end{block}
\end{frame}

	% Slide 14
	\begin{frame}{Common Deductive Fallacies: When Logic Goes Wrong}
		\begin{itemize}
			\item Even when attempting deductive reasoning, people often make logical errors called \textbf{fallacies}.
			\item \textbf{Affirming the consequent}: If P then Q, Q, Therefore P—this reverses the logic incorrectly.
			\item \textbf{Denying the antecedent}: If P then Q, Not P, Therefore Not Q—this also breaks the logical connection.
			\item Recognizing these patterns helps you spot invalid arguments that might seem convincing at first glance.
		\end{itemize}
		
		\begin{block}{Common Invalid Forms}
			\textbf{Affirming the Consequent:} If it's raining, the ground is wet. The ground is wet. Therefore, it's raining.
			\\[0.5em]
			\textbf{Denying the Antecedent:} If it's raining, the ground is wet. It's not raining. Therefore, the ground is not wet.
		\end{block}
	\end{frame}

% Slide 12
\begin{frame}{Sound Arguments: When Valid Logic Meets True Premises}
	\begin{itemize}
		\item A \textbf{sound} argument is a valid deductive argument where all the premises are actually true in reality.
		\item Sound arguments give us the strongest possible support for their conclusions—the conclusion must be true.
		\item Many deductive arguments we encounter are valid but not sound because one or more premises are false or questionable.
		\item Distinguishing between validity (logical structure) and soundness (logical structure + true premises) is crucial for critical thinking.
	\end{itemize}
	
	\begin{alertblock}{Sound vs. Valid}
		\textbf{Valid:} The logic works correctly (if premises were true, conclusion would follow)
		\\[0.5em]
		\textbf{Sound:} The logic works correctly AND the premises are actually true
	\end{alertblock}
\end{frame}

% Additional Slide: Valid but Unsound Argument
\begin{frame}{Valid but Unsound: Veronica Mars's Logical but False Reasoning}
	\begin{itemize}
		\item Here's a valid argument with false premises: "All high school students cheat on tests. Logan is a high school student. Therefore, Logan cheats on tests."
		\item The argument is \textbf{valid} because if the premises were true, the conclusion would necessarily follow.
		\item However, the argument is \textbf{unsound} because the first premise ("All high school students cheat on tests") is clearly false.
		\item This shows why both logical structure (validity) and factual accuracy (true premises) matter for strong deductive arguments.
	\end{itemize}
	
	\begin{example}
		\scriptsize
		\textbf{Analysis of Veronica's argument:}
		\begin{itemize}
			\item \textit{Premise 1:} All high school students cheat on tests. (False)
			\item \textit{Premise 2:} Logan is a high school student. (True)
			\item \textit{Conclusion:} Logan cheats on tests. (False conclusion from false premise)
			\item \textit{Verdict:} Valid form but unsound due to false premise
		\end{itemize}
		\end{example}
\end{frame}

% Slide 15
\begin{frame}{Inductive Arguments: Reasoning from Evidence to Likelihood}
	\begin{itemize}
		\item An \textbf{inductive argument} moves from specific observations to general conclusions that are probably, but not necessarily, true.
		\item Unlike deductive arguments, inductive arguments acknowledge that new evidence could change our conclusions.
		\item We evaluate inductive arguments as \textbf{strong} (premises provide good support) or \textbf{weak} (premises provide poor support).
		\item The conclusion of even a strong inductive argument could turn out to be false—that's the nature of reasoning about probabilities.
	\end{itemize}
	
	\begin{block}{Definition: Inductive Argument}
		An argument where the premises are intended to provide probable support for the conclusion. If the premises are true and we've considered all relevant evidence, the conclusion is likely to be true.
	\end{block}
\end{frame}

% Slide 16
\begin{frame}{Strong vs. Weak: Measuring Inductive Support}
	\begin{itemize}
		\item A \textbf{strong} inductive argument is one where true premises make the conclusion very likely to be true.
		\item A \textbf{weak} inductive argument is one where true premises provide little support for the conclusion.
		\item Strength comes in degrees—arguments can be stronger or weaker rather than simply strong or weak.
		\item Unlike validity, inductive strength depends partly on what we know about the world, not just logical structure.
	\end{itemize}
	
	\begin{example}
		\textbf{Strong:} In 95\% of cases with this DNA evidence pattern, the suspect is guilty. We have this DNA pattern. Therefore, the suspect is probably guilty.
		\\[0.5em]
		\textbf{Weak:} The suspect wore a red shirt. Some criminals wear red shirts. Therefore, the suspect is probably guilty.
	\end{example}
	\end{frame}

% Slide 17
\begin{frame}{Common Inductive Argument Forms Table}
	\begin{itemize}
		\item Inductive arguments take several common forms, each with specific patterns and evaluation criteria.
		\item These forms help us reason from limited observations to broader conclusions about patterns and probabilities.
		\item Unlike deductive forms, inductive forms don't guarantee their conclusions but aim to make them highly probable.
		\item Understanding these forms helps identify when inductive reasoning is being used and how strong it is.
	\end{itemize}
	
	\begin{table}[h]
		\scriptsize
		\centering
		\begin{tabular}{|l|l|}
			\hline
			\textbf{Argument Form} & \textbf{Structure} \\
			\hline
			Generalization & Sample has property X; Therefore, population has X \\
			\hline
			Prediction & Pattern held in past; Therefore, pattern will continue \\
			\hline
			Argument from Analogy & A and B are similar; A has property X; Therefore, B has X \\
			\hline
			Argument from Authority & Expert E claims P; Therefore, P is probably true \\
			\hline
		\end{tabular}
	\end{table}
\end{frame}

% Slide 18
\begin{frame}{The Total Evidence Requirement}
	\begin{itemize}
		\item The \textbf{total evidence requirement} states that inductive arguments must consider all relevant available evidence, not just supporting evidence.
		\item For example, a detective might seem to have a strong case against the suspect based on fingerprints and motive.
		\item However, this argument would be much weaker if crucial security camera footage provided the suspect with an alibi.
		\item Ignoring contrary evidence or cherry-picking data makes even seemingly strong inductive arguments unreliable.
	\end{itemize}
	
	\begin{alertblock}{The Total Evidence Requirement}
		An inductive argument is only as strong as its weakest ignored piece of relevant evidence. All available relevant evidence must be considered for a fair assessment.
	\end{alertblock}
\end{frame}

% Slide 19

\begin{frame}{Generalizations: Officer Williams and the Crime Pattern}
	\begin{itemize}
		\item \textbf{Inductive generalizations} move from observations about a sample to conclusions about a larger population.
		\item Example: Officer Williams notices that 80\% of burglaries in her district occur between 2-4 PM on weekdays when residents are at work.
		\item Based on this pattern, she schedules extra patrols during these hours and sees a 30\% reduction in break-ins.
		\item The strength of the generalization depends on sample size, representativeness, and the margin of the observed pattern.
	\end{itemize}
	
	\begin{example}
		\textbf{Williams's reasoning:}
		\begin{itemize}
			\item \textit{Sample:} 200 burglaries over 6 months, 160 occurred 2-4 PM on weekdays
			\item \textit{Generalization:} Most future burglaries will occur 2-4 PM on weekdays
			\item \textit{Action:} Increase patrols during high-risk hours
		\end{itemize}
	\end{example}
	\end{frame}

% Slide 20
\begin{frame}{Predictions: Forensic Analyst Dr. Kim's Weather Case}
	\begin{itemize}
		\item \textbf{Inductive predictions} use past patterns to forecast future events, acknowledging that patterns might change.
		\item Example: Dr. Kim analyzes how weather affects evidence preservation at outdoor crime scenes over several years.
		\item She discovers that DNA evidence degrades 50\% faster in temperatures above 85°F with high humidity.
		\item Based on weather forecasts, she now prioritizes evidence collection and adjusts testing procedures accordingly.
	\end{itemize}
	
	\begin{example}
		\scriptsize
		\textbf{Dr. Kim's prediction model:}
		\begin{itemize}
			\item \textit{Past pattern:} Hot, humid weather significantly degrades DNA evidence
			\item \textit{Current situation:} Weather forecast shows 90°F and 80\% humidity
			\item \textit{Prediction:} DNA evidence will degrade rapidly at this crime scene
			\item \textit{Response:} Expedite evidence collection and use enhanced preservation methods
		\end{itemize}
	\end{example}
\end{frame}	

% Slide 21
\begin{frame}{Arguments from Analogy: How Detective Foster Solved the Cold Case}
	\begin{itemize}
		\item \textbf{Arguments from analogy} reason that if two things are similar in known ways, they're likely similar in unknown ways.
		\item Example: Detective Foster reopened a 10-year-old murder case and noticed striking similarities to a recent solved case.
		\item Both victims were similar age, profession, and found in similar locations with identical evidence patterns.
		\item Foster reasoned that the same perpetrator likely committed both crimes and focused her investigation accordingly.
	\end{itemize}
	
	\begin{example}
		\scriptsize
		\textbf{Foster's analogical reasoning:}
		\begin{itemize}
			\item \textit{Case A (solved):} Female lawyer, 35, found in park, specific knife wounds, no robbery
			\item \textit{Case B (unsolved):} Female lawyer, 34, found in park, identical wounds, no robbery
			\item \textit{Conclusion:} Same perpetrator likely committed both crimes
			\item \textit{Result:} DNA from Case A led to arrest in Case B
		\end{itemize}
	\end{example}
\end{frame}

% Slide 22
\begin{frame}{Arguments from Authority: When Expert Testimony Matters}
	\begin{itemize}
		\item \textbf{Arguments from authority} accept a conclusion because a credible expert or authority figure endorses it.
		\item These arguments can be strong when the authority is truly expert in the relevant field and there's expert consensus.
		\item However, they're weak when the authority lacks relevant expertise, has conflicts of interest, or experts disagree.
		\item In legal and scinetific settings, expert testimony often provides crucial inductive support for conclusions about complex technical matters.
	\end{itemize}
	
	\begin{block}{Evaluating Arguments from Authority}
		\textbf{Strong when:} True expert, relevant field, expert consensus, no conflicts of interest
		\\[0.5em]
		\textbf{Weak when:} False expert, irrelevant field, expert disagreement, conflicts of interest
	\end{block}
\end{frame}

% Slide 23
\begin{frame}{Sample Size and Representativeness: Detective Johnson's Survey}
	\begin{itemize}
		\item The strength of inductive generalizations depends heavily on \textbf{sample size} and \textbf{representativeness}.
		\item Example: Detective Johnson surveyed community members about drug activity but initially only interviewed people during business hours.
		\item This gave her a biased sample of mostly retirees and unemployed residents, missing working people's perspectives.
		\item After expanding her survey times and methods, she got a more representative sample and very different results.
	\end{itemize}
	
	\begin{example}
		\scriptsize
		\textbf{Johnson's sampling lesson:}
		\begin{itemize}
			\item \textit{Initial sample:} 50 people, 9 AM-5 PM, mostly elderly/unemployed
			\item \textit{Result:} 80\% said no drug problem in neighborhood
			\item \textit{Expanded sample:} 200 people, various times, all demographics  
			\item \textit{New result:} 60\% said significant drug problem exists
		\end{itemize}
	\end{example}
\end{frame}

% Slide 24
\begin{frame}{Evaluating Inductive Strength: A Step-by-Step Guide}
	\begin{itemize}
		\item Evaluating inductive arguments requires checking multiple factors that affect how well the premises support the conclusion.
		\item Start by identifying the type of inductive argument (generalization, prediction, analogy, or authority).
		\item Check whether all relevant evidence has been considered and whether the sample or comparison is appropriate.
		\item Consider alternative explanations and assess whether the conclusion is the most likely given the evidence.
	\end{itemize}
	
	\begin{block}{Inductive Evaluation Checklist}
		\begin{enumerate}
			\item What type of inductive argument is this?
			\item Has all relevant evidence been considered?
			\item Is the sample size adequate and representative?
			\item Are there alternative explanations for the evidence?
			\item How probable is the conclusion given the premises?
		\end{enumerate}
	\end{block}
\end{frame}

% Slide 25
\begin{frame}{Inspector Garcia and the Probability Assessment}
	\begin{itemize}
		\item Example: Inspector Garcia must weigh conflicting evidence in a robbery case to assess the probability of the suspect's guilt.
		\item The fingerprint evidence strongly suggests guilt (found at 90\% of crime scenes where perpetrator identified).
		\item However, the suspect has a solid alibi confirmed by three independent witnesses with no motive to lie.
		\item Garcia concludes the evidence is roughly balanced, requiring additional investigation before making an arrest.
	\end{itemize}
	
	\begin{example}
		\scriptsize
		\textbf{Garcia's probability assessment:}
		\begin{itemize}
			\item \textit{Supporting evidence:} Fingerprints (90\% reliability), suspicious behavior
			\item \textit{Contradicting evidence:} Solid alibi, no clear motive, good character references
			\item \textit{Conclusion:} Evidence insufficient for high confidence in guilt
			\item \textit{Action:} Continue investigation before proceeding
		\end{itemize}
	\end{example}
\end{frame}

% Slide 26
\begin{frame}{Common Inductive Fallacies: When Evidence Misleads}
	\begin{itemize}
		\item \textbf{Hasty generalization} draws broad conclusions from insufficient or unrepresentative samples.
		\item \textbf{False analogy} compares things that are too different in relevant respects to support the conclusion.
		\item \textbf{Appeal to inappropriate authority} relies on expertise from the wrong field or biased sources.
		\item \textbf{Cherry-picking} selects only supporting evidence while ignoring contradictory data.
	\end{itemize}
	
	\begin{alertblock}{Warning Signs of Weak Inductive Arguments}
		Small sample sizes, biased samples, irrelevant comparisons, inappropriate experts, ignored contrary evidence, or conclusions much stronger than the evidence warrants.
	\end{alertblock}
\end{frame}

% Slide 27
\begin{frame}{Cogent Arguments: When Strong Logic Meets True Premises}
	\begin{itemize}
		\item A \textbf{cogent} argument is a strong inductive argument where all the premises are actually true in reality.
		\item Just as soundness is the gold standard for deductive arguments, cogency is the gold standard for inductive arguments.
		\item Cogent arguments give us the best possible inductive support for their conclusions—the conclusion is very likely to be true.
		\item Many inductive arguments we encounter are strong but not cogent because one or more premises are false or questionable.
	\end{itemize}
	
	\begin{alertblock}{Cogent vs. Strong}
		\textbf{Strong:} The logic works well (if premises were true, conclusion would be very likely)
		\\[0.5em]
		\textbf{Cogent:} The logic works well AND the premises are actually true
	\end{alertblock}
\end{frame}

% Slide 28
\begin{frame}{Abductive Arguments: Inference to the Best Explanation}
	\begin{itemize}
		\item \textbf{Abductive arguments} (also called inference to the best explanation) start with puzzling observations and conclude that a particular explanation is most likely correct.
		\item Unlike deductive arguments (which guarantee conclusions) or inductive arguments (which assess probability), abductive arguments compare competing explanations.
		\item We evaluate abductive arguments as \textbf{better} or \textbf{worse} based on how well they explain the available evidence.
		\item Scientists, doctors, and detectives constantly use abductive reasoning to form hypotheses and diagnoses.
	\end{itemize}
	
	\begin{block}{Definition: Abductive Argument}
		An argument that concludes a particular explanation is the best available account of some puzzling phenomenon or set of observations.
	\end{block}
\end{frame}

% Slide 28
\begin{frame}{Better vs. Worse Explanations: The Criteria}
	\begin{itemize}
		\item A \textbf{better explanation} accounts for more of the evidence, makes fewer unsupported assumptions, and fits with our background knowledge.
		\item \textbf{Explanatory scope}: How much of the evidence does the explanation account for?
		\item \textbf{Simplicity}: Does the explanation avoid unnecessary complexity and assumptions?
		\item \textbf{Consistency}: Does the explanation fit with other things we know to be true?
	\end{itemize}
	
	\begin{block}{Criteria for Better Explanations}
		\begin{itemize}
			\item \textbf{Scope:} Explains more phenomena
			\item \textbf{Simplicity:} Fewer assumptions, less complexity  
			\item \textbf{Consistency:} Fits with background knowledge
			\item \textbf{Testability:} Makes predictions we can check
			\item \textbf{Precision:} Specific rather than vague
		\end{itemize}
	\end{block}
\end{frame}

% Slide 29
\begin{frame}{Common Abductive Reasoning Patterns Table}
	\begin{itemize}
		\item Abductive reasoning appears in various forms across different fields, from medical diagnosis to criminal investigation.
		\item Each pattern involves comparing multiple possible explanations for observed phenomena.
		\item The goal is identifying which explanation best accounts for all available evidence with the fewest problematic assumptions.
		\item These patterns help structure our thinking when facing complex, puzzling situations that require explanatory hypotheses.
	\end{itemize}
	
	\begin{table}[h]
		\scriptsize
		\centering
		\begin{tabular}{|l|l|}
			\hline
			\textbf{Reasoning Pattern} & \textbf{Structure} \\
			\hline
			Medical Diagnosis & Symptoms X, Y, Z; Disease A best explains X, Y, Z \\
			\hline
			Criminal Investigation & Evidence A, B, C; Suspect theory best explains A, B, C \\
			\hline
			Scientific Hypothesis & Observations P, Q, R; Theory T best explains P, Q, R \\
			\hline
			Technical Troubleshooting & Problems X, Y; Cause C best explains X, Y \\
			\hline
		\end{tabular}
	\end{table}
\end{frame}

% Slide 30
\begin{frame}{Dr. House's Diagnostic Method: Medical Abduction}
	\begin{itemize}
		\item Dr. House faces a patient with fever, joint pain, skin rash, and neurological symptoms—a puzzling combination.
		\item He considers multiple diagnoses: lupus, Lyme disease, multiple sclerosis, and a rare autoimmune condition.
		\item The rare autoimmune condition explains all four symptoms, while other diagnoses only explain some symptoms well.
		\item House concludes this is the best explanation despite its rarity, and targeted tests confirm the diagnosis.
	\end{itemize}
	
	\begin{example}
		\scriptsize
		\textbf{House's diagnostic reasoning:}
		\begin{itemize}
			\item \textit{Lupus:} Explains fever, joint pain, rash but not neurological symptoms
			\item \textit{Lyme disease:} Explains fever, joint pain, some neurological symptoms but rash is wrong type
			\item \textit{Rare autoimmune:} Explains all four symptoms perfectly, fits patient's history
			\item \textit{Conclusion:} Rare autoimmune condition is the best explanation
		\end{itemize}
	\end{example}
\end{frame}

% Slide 31
\begin{frame}{Detective Poirot's Explanation Evaluation}
	\begin{itemize}
		\item Poirot investigates a locked-room murder where the victim was found alone in a study with the door locked from inside.
		\item \textbf{Suicide theory}: Simple but doesn't explain the missing gun or the victim's left-handed wound (victim was right-handed).
		\item \textbf{Secret passage theory}: Explains the locked room but no evidence of hidden entrances after thorough searches.
		\item \textbf{Accomplice theory}: The butler unlocked the door after the murder, explains all evidence with minimal assumptions.
	\end{itemize}
	
	\begin{block}{Poirot's Comparison}
		\scriptsize
		\begin{itemize}
			\item \textbf{Suicide:} Simple but fails to explain key evidence
			\item \textbf{Secret passage:} Explains locked room but unsupported by physical evidence
			\item \textbf{Accomplice:} Explains all evidence, requires only one reasonable assumption
		\end{itemize}
	\end{block}
\end{frame}

% Slide 32
\begin{frame}{Scientific Explanations: Forensic Scientist Dr. Lee's Discovery}
	\begin{itemize}
		\item Dr. Lee analyzes unusual blood spatter patterns that don't match typical gunshot or stabbing scenarios.
		\item \textbf{Blunt force theory}: Doesn't explain the fine droplet pattern or the circular distribution.
		\item \textbf{Explosion theory}: Explains the droplet size and distribution but no chemical residue found.
		\item \textbf{High-velocity impact theory}: Best explains the pattern—victim was struck by a vehicle, then moved indoors.
	\end{itemize}
	
	\begin{example}
		\scriptsize
		\textbf{Dr. Lee's scientific reasoning:}
		\begin{itemize}
			\item \textit{Observation:} Fine blood droplets in circular pattern, no typical weapon signatures
			\item \textit{Vehicle impact theory:} Explains droplet size, pattern, and secondary transfer
			\item \textit{Prediction:} Should find vehicle paint or glass fragments
			\item \textit{Confirmation:} Microscopic analysis reveals automotive paint particles
		\end{itemize}
	\end{example}
\end{frame}


% Slide 33
\begin{frame}{Competing Hypotheses: How Agent Cooper Weighs Evidence}
	\begin{itemize}
		\item Agent Cooper investigates corporate fraud with evidence pointing toward two equally viable suspect theories.
		\item \textbf{Inside job theory}: CFO had access, motive (gambling debts), and opportunity during the audit period.
		\item \textbf{Cyber attack theory}: External hackers exploited security vulnerabilities, similar to recent attacks on other companies.
		\item Cooper must carefully weigh which explanation better accounts for the digital forensics, timing, and financial patterns.
	\end{itemize}
	
	\begin{block}{Cooper's Evidence Analysis}
		\scriptsize
		\begin{itemize}
			\item \textbf{Supporting inside job:} Access logs, financial pressure, specific knowledge required
			\item \textbf{Supporting cyber attack:} Sophisticated methods, international IP addresses, similar recent cases
			\item \textbf{Problem:} Both theories explain the evidence reasonably well
			\item \textbf{Solution:} Gather additional discriminating evidence
		\end{itemize}
	\end{block}
\end{frame}

% Slide 34
\begin{frame}{The Limits of Abductive Reasoning}
	\begin{itemize}
		\item Abductive reasoning gives us the \textbf{best available explanation}, not necessarily the \textbf{true explanation}.
		\item Even the best explanation might be wrong if we lack crucial evidence or haven't considered the right alternative.
		\item Example: Scientists, detectives, etc. may know that evidence strongly suggests certain conclusions but doesn't provide absolute certainty.
		\item Abductive conclusions should be held tentatively and revised when new evidence emerges.
	\end{itemize}
	
	\begin{alertblock}{Abductive Limitations}
		The best explanation available now may not be the best explanation tomorrow. New evidence can dramatically change which explanation seems most plausible.
	\end{alertblock}
\end{frame}

% Slide 35
\begin{frame}{Comparing the Three Argument Types: A Summary Table}
	\begin{itemize}
		\item Each argument type serves different purposes and requires different evaluation standards.
		\item Understanding when to use each type helps you reason more effectively in different situations.
		\item Many complex arguments combine all three types, using each where it's most appropriate.
		\item Critical thinking requires recognizing which type of argument is being used and applying the correct evaluation criteria.
	\end{itemize}
	
	\begin{table}[h]
		\centering
		\small
		\begin{tabular}{|l|l|l|l|}
			\hline
			\textbf{Type} & \textbf{Goal} & \textbf{Evaluation} & \textbf{Strength} \\
			\hline
			Deductive & Certainty & Valid/Invalid & Logical necessity \\
			\hline
			Inductive & Probability & Strong/Weak & Likely conclusions \\
			\hline
			Abductive & Best explanation & Better/Worse & Explanatory power \\
			\hline
		\end{tabular}
	\end{table}
\end{frame}

% Slide 36
\begin{frame}{The Complete Investigator: When to Use Each Type of Reasoning}
	\begin{itemize}
		\item Expert investigators combine all three reasoning types strategically throughout their cases.
		\item Use \textbf{deductive reasoning} when you have clear rules or principles that apply directly to your situation.
		\item Use \textbf{inductive reasoning} when you need to predict future events or generalize from observed patterns.
		\item Use \textbf{abductive reasoning} when you face puzzling evidence that needs explanation or when comparing competing theories.
	\end{itemize}
	
	\begin{example}
		\scriptsize
		\textbf{Complete investigation approach:}
		\begin{itemize}
			\item \textit{Deductive:} "If the suspect was at location X, then he couldn't have committed the crime at location Y."
			\item \textit{Inductive:} "Based on similar cases, this type of crime usually involves someone the victim knew."
			\item \textit{Abductive:} "The best explanation for all this evidence is that the business partner committed the murder."
		\end{itemize}
	\end{example}
\end{frame}

\end{document}